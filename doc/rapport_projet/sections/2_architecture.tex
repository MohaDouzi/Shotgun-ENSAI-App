
\chapter{Architecture générale}

\chapter{Architecture générale}

\section{Introduction}

L'application a été conçue suivant une architecture modulaire et robuste, inspirée des approches couramment utilisées dans l’industrie logicielle.
Le système repose sur une séparation claire entre :

\begin{itemize}
    \item les \textbf{vues} (interface en console),
    \item les \textbf{services métier} (logique applicative),
    \item les \textbf{DAO} (accès aux données),
    \item la \textbf{base de données PostgreSQL},
    \item des \textbf{composants externes} : envoi d’e-mails via l'API Brevo.
\end{itemize}

Cette structure a été choisie pour rendre l’application évolutive, testable et maintenable, tout en permettant d’ajouter facilement de nouvelles fonctionnalités (statistiques, rappels e-mail, front-end dans le futur, etc.).

\section{Vue d’ensemble}

La Figure~\ref{fig:uml-global} présente l’architecture générale de l’application.
Elle illustre les interactions entre les différentes couches :

\begin{itemize}
    \item la \textbf{Vue} reçoit les actions de l'utilisateur ;
    \item la \textbf{Vue} appelle un \textbf{Service} correspondant ;
    \item le \textbf{Service} applique la logique métier ;
    \item le \textbf{Service} utilise un \textbf{DAO} pour lire/écrire dans PostgreSQL ;
    \item certains Services communiquent avec l’API \textbf{Brevo} ;
    \item la \textbf{Base de données} assure persistance, cohérence et intégrité.
\end{itemize}

\begin{figure}[H]
\centering
\includegraphics[width=0.9\textwidth]{figures/uml_global.png}
\caption{Architecture générale de l'application}
\label{fig:uml-global}
\end{figure}

\section{Organisation en couches}

\subsection{Couches de présentation : les Vues}

Les vues correspondent aux interfaces console de l'application.
Elles sont centralisées dans le répertoire \texttt{view/}.
Leur rôle est strictement limité à :

\begin{itemize}
    \item afficher des informations à l’utilisateur,
    \item recueillir les valeurs saisies via InquirerPy (menus interactifs),
    \item appeler le service approprié,
    \item rediriger vers une autre vue après une action.
\end{itemize}

Les vues ne contiennent aucune logique métier, ce qui assure leur simplicité et leur testabilité.

Par exemple :

\begin{itemize}
    \item \texttt{creer\_evenement\_vue.py} permet à un administrateur de créer un événement ;
    \item \texttt{reservation\_vue.py} gère l’inscription d’un étudiant ;
    \item \texttt{mes\_reservations\_vue.py} affiche les réservations personnelles ;
    \item \texttt{modifier\_evenement\_vue.py} permet de mettre à jour un événement existant.
\end{itemize}

\subsection{Couche métier : les Services}

La couche service encapsule toute la \textbf{logique métier}.
Son rôle est essentiel car elle :

\begin{itemize}
    \item applique les règles internes (capacités des bus, unicité des réservations, formats des entrées, cohérence),
    \item délègue l’accès aux données au DAO,
    \item communique avec les APIs externes,
    \item transforme les données pour les envoyer ou les afficher.
\end{itemize}

Chaque domaine possède son service :

\begin{itemize}
    \item \textbf{EvenementService} : création, mise à jour, listage, suppression.
    \item \textbf{ReservationService} : gestion des inscriptions, code unique, envoi du mail de confirmation.
    \item \textbf{ParticipantService} : comptes utilisateurs, authentification.
    \item \textbf{BusService} : configuration et gestion des transports liés à un événement.
\end{itemize}

La logique métier reste ainsi parfaitement isolée et réutilisable.

\subsection{Couche DAO : accès aux données}

Les DAO (\textit{Data Access Objects}) exécutent les requêtes SQL et manipulent les objets de la base.
Ils sont responsables de :

\begin{itemize}
    \item l’exécution de requêtes SQL PostgreSQL,
    \item la conversion en objets Pydantic à la sortie,
    \item la gestion de \texttt{commit/rollback},
    \item la protection contre les injections SQL (via requêtes paramétrées).
\end{itemize}

Chaque table possède son DAO :

\begin{itemize}
    \item EvenementDao ;
    \item ReservationDao ;
    \item ParticipantDao ;
    \item BusDao.
\end{itemize}

Le respect de la séparation \textbf{Service ↔ DAO} garantit une base solide pour l'évolution de l'application.

\section{Structure de la base de données}

La base est composée de plusieurs tables principales :

\begin{itemize}
    \item \textbf{utilisateur} : participants et administrateurs ;
    \item \textbf{evenement} : description complète d’un événement ;
    \item \textbf{transport} : informations spécifiques aux bus aller/retour ;
    \item \textbf{reservation} : inscription d'un utilisateur à un événement, options choisies, code unique.
\end{itemize}

La modélisation suit les bonnes pratiques enseignées en 1A (formes normales) :
pas de redondance, clés primaires artificielles, correspondances via clés étrangères, cohérence des contraintes.

\section{Sécurité de l'application}

L'application intègre plusieurs mécanismes de sécurité :

\begin{itemize}
    \item \textbf{Hashage des mots de passe} via bcrypt ;
    \item \textbf{Isolation de la logique métier} dans les services ;
    \item \textbf{Vérification systématique des rôles} (admin / participant) dans les vues ;
    \item \textbf{Génération de codes uniques} pour chaque réservation ;
    \item \textbf{Utilisation d’une API d’envoi d’e-mails} fiable et reconnue (Brevo) ;
    \item \textbf{Requêtes SQL paramétrées} contre les injections.
\end{itemize}

L'ensemble respecte les bonnes pratiques d’une application universitaire et professionnelle.

\section{Interaction avec l'API Brevo}

Certaines fonctionnalités, notamment :

\begin{itemize}
    \item envoi du mail de confirmation d’inscription,
    \item envoi du mail de modification de réservation,
    \item envoi du mail d’annonce d’un nouvel événement,
\end{itemize}

nécessitent un accès externe via HTTP.

Le service \texttt{email\_service} communique avec Brevo en envoyant un JSON contenant :

\begin{itemize}
    \item le destinataire,
    \item le sujet,
    \item le texte brut et/ou HTML,
    \item la signature du BDE,
    \item la clé API (dans le fichier \texttt{.env}).
\end{itemize}

\section{Synthèse}

L'architecture respecte les principes de séparation des responsabilités et garantit :

\begin{itemize}
    \item une maintenance facilitée,
    \item une évolutivité naturelle,
    \item une robustesse vis-à-vis des erreurs,
    \item une compatibilité immédiate avec un éventuel futur front-end.
\end{itemize}

Le chapitre suivant décrit en détail un processus central : l’inscription à un événement.
