
\chapter{Contexte de l'application}

\chapter{Contexte de l'application}

\section{Introduction générale}

La vie associative de l’ENSAI repose en partie sur l'organisation régulière
d’événements étudiants tels que le Week-End d’Intégration (WEI), le Gala,
les soirées étudiantes ou encore certaines sorties culturelles.
Ces événements attirent chaque année plusieurs centaines d’étudiants.
Toutefois, leur organisation logistique repose souvent sur des outils
hétérogènes : formulaires Google, fichiers Excel partagés, inscriptions manuelles,
ou encore des échanges informels via les réseaux sociaux.

Ce mode de fonctionnement présente plusieurs limites :

\begin{itemize}
    \item \textbf{pas de suivi automatisé du nombre de places restantes ;}
    \item \textbf{un risque d’erreurs élevé} (doublons, oublis, mauvaise saisie) ;
    \item \textbf{une charge mentale importante pour les membres du BDE} ;
    \item \textbf{aucune traçabilité centralisée} des réservations ;
    \item \textbf{difficulté à envoyer des rappels ou confirmations personnalisées}.
\end{itemize}

Face à ces constats, il devient pertinent de mettre en place une application
informatique centralisée, robuste et automatisée pour gérer l’intégralité du
processus d’inscription à un événement.

\section{Objectifs du projet}

L’objectif du projet est de développer une application en Python permettant :

\begin{enumerate}
    \item la \textbf{gestion complète des événements} par les administrateurs ;
    \item la \textbf{gestion des inscriptions} par les étudiants ;
    \item la \textbf{gestion centralisée de la capacité des bus} aller et retour ;
    \item la \textbf{génération automatique d’un code de réservation} sécurisé ;
    \item l’\textbf{envoi automatisé d’e-mails via l’API Brevo} ;
    \item la \textbf{visualisation en temps réel} des inscrits par évènement ;
    \item la \textbf{modification et suppression} d’une réservation existante.
\end{enumerate}

Ces fonctionnalités permettent de couvrir l'ensemble des besoins opérationnels
du BDE pour un événement classique, tout en améliorant l’expérience utilisateur
des participants.

\section{Utilisateurs de l'application}

Le système gère deux types d’utilisateurs, chacun ayant un rôle distinct
dans l’application :

\begin{itemize}
    \item \textbf{Les administrateurs} (membres du BDE) :
    Ils créent les événements, configurent les bus, suivent les inscriptions,
    gèrent les capacités, et peuvent modifier ou supprimer n'importe quelle
    réservation.

    \item \textbf{Les participants} (étudiants inscrits à l’ENSAI) :
    Ils consultent les événements ouverts aux inscriptions, réservent un créneau
    bus, reçoivent un e-mail de confirmation et peuvent modifier leur réservation.
\end{itemize}

Cette distinction est directement reflétée dans la base de données (champ
\texttt{administrateur} dans la table \texttt{utilisateur}), dans les vues Python
et dans les contrôles d’accès.

\section{Données manipulées par l'application}

La solution repose sur plusieurs familles de données :

\subsection{Données internes}

\begin{itemize}
    \item informations sur les utilisateurs : nom, prénom, e-mail, mot de passe ;
    \item informations sur les événements : titre, lieu, date, capacité globale ;
    \item données de transport : nombre de places pour chaque bus aller/retour ;
    \item réservations : date d’inscription, options choisies, code unique ;
    \item suivi statistique : nombre d’inscrits, taux de remplissage par bus.
\end{itemize}

Toutes ces données sont stockées dans une base PostgreSQL normalisée,
garantissant intégrité, cohérence et admissibilité scolaire (1NF, 2NF, 3NF).

\subsection{Données externes (API Brevo)}

L'application intègre un service externe : l'API d’envoi d’e-mails Brevo
(précédemment SendinBlue).
Cette API permet :

\begin{itemize}
    \item l'envoi automatique d'e-mails de confirmation ;
    \item l’envoi d’e-mails d’alerte lors de la création d’un nouvel événement ;
    \item l’envoi potentiel de relances de paiement (fonctionnalité optionnelle).
\end{itemize}

Les données envoyées à Brevo suivent un schéma JSON défini : destinataire,
objet, contenu textuel, contenu HTML, nom de l’expéditeur, etc.
Aucune donnée sensible n’est transmise (les mots de passe restent internes).

\section{Synthèse}

Cette application répond à un besoin réel du BDE et constitue une solution
professionnelle pour gérer des événements complexes.

Les problématiques couvertes incluent :

\begin{itemize}
    \item la gestion multi-utilisateurs avec rôles ;
    \item la gestion des capacités en temps réel ;
    \item la communication automatisée ;
    \item la sécurité (mot de passe, cohérence DB, code unique) ;
    \item la scalabilité : gestion de centaines d’inscriptions.
\end{itemize}

Le chapitre suivant présente l’architecture logicielle permettant de
mettre en œuvre ces fonctionnalités.
