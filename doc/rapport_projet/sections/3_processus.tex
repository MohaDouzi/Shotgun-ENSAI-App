
\chapter{Fonctionnement de l'application}

\chapter{Fonctionnement de l'application}

\section{Introduction}

Ce chapitre présente le fonctionnement interne de l’application, en détaillant le cheminement d’un utilisateur lors des principales opérations.
L’objectif est de montrer comment l’ensemble des composants décrits dans le chapitre précédent interagissent pour satisfaire les besoins exprimés.

Nous consacrons une première partie à l’étude approfondie d’un processus central : \textbf{l’inscription d’un participant à un événement}.
Nous analysons ensuite les autres fonctionnalités majeures afin de présenter la logique générale sans toutefois entrer dans un détail exhaustif de toutes les méthodes Python.

\section{Processus central : inscription d'un participant à un événement}

L’inscription à un événement constitue l’élément pivot de l’application.
Ce processus implique :

\begin{itemize}
    \item l’utilisateur (participant),
    \item le système d’authentification,
    \item la vue d’inscription,
    \item les services \textbf{EvenementService}, \textbf{BusService} et \textbf{ReservationService},
    \item la base PostgreSQL,
    \item l’API Brevo pour l’envoi d’un e-mail de confirmation.
\end{itemize}

Les différentes étapes sont décrites ci-après.

\subsection{Étape 1 : authentification du participant}

À son arrivée dans l’application, un participant doit s’identifier via la vue \texttt{ConnexionClientVue}.
Cette vue utilise le \textbf{ParticipantService}, lequel s'appuie sur le \textbf{ParticipantDao} pour vérifier :

\begin{enumerate}
    \item que l’adresse e-mail existe ;
    \item que le mot de passe saisi correspond au hash stocké (bcrypt) ;
    \item que l’utilisateur n’est pas administrateur (séparation stricte des rôles).
\end{enumerate}

En cas de succès, un objet \texttt{Session} est créé pour stocker le participant connecté.

\subsection{Étape 2 : choix d’un événement}

La vue \texttt{ReservationVue} appelle :

\begin{itemize}
    \item \texttt{EvenementService.list\_events\_with\_places()},
\end{itemize}

qui retourne la liste des événements ouverts aux inscriptions ainsi que leurs places restantes.
Ce calcul repose sur une requête SQL optimisée utilisant un \texttt{LEFT JOIN} sur les réservations existantes.

Le participant sélectionne un événement.
S’il n’y a plus de places, l’application refuse immédiatement l’inscription.

\subsection{Étape 3 : sélection des créneaux de bus}

Une fois l’événement choisi, l’application affiche :

\begin{itemize}
    \item la liste des bus \textbf{ALLER} associés à l'événement ;
    \item la liste des bus \textbf{RETOUR}.
\end{itemize}

Chaque créneau possède :

\begin{itemize}
    \item une description (horaire, lieu de départ),
    \item une capacité maximale,
    \item le nombre de places restantes,
    \item un identifiant unique.
\end{itemize}

Le participant choisit indépendamment un bus aller et un bus retour (ou aucun).

\subsection{Étape 4 : options supplémentaires}

Le système propose ensuite:

\begin{itemize}
    \item option \textbf{Adhérent BDE} ;
    \item option \textbf{SAM} ;
    \item option \textbf{Boisson}.
\end{itemize}

Cela permet d’adapter le prix final et de fournir au BDE des informations utiles à l’organisation.

\subsection{Étape 5 : création de la réservation}

La vue appelle ensuite :

\begin{center}
\texttt{ReservationService.create\_reservation(...)}
\end{center}

Le service effectue les contrôles suivants :

\begin{enumerate}
    \item L’utilisateur n’a pas déjà une réservation pour cet événement.
    \item Les bus choisis ne sont pas complets.
    \item Les données sont cohérentes (types, contraintes de capacité).
\end{enumerate}

Une fois validé, il demande au \textbf{DAO} :

\begin{itemize}
    \item d’insérer la réservation dans PostgreSQL ;
    \item de générer un \textbf{code de réservation unique} (hash alphanumérique).
\end{itemize}

L’objet \texttt{ReservationModelOut} est renvoyé à la vue.

\subsection{Étape 6 : envoi du mail de confirmation (Brevo)}

Enfin, le service envoie automatiquement un e-mail (via \texttt{api\_brevo.py}) :

\begin{itemize}
    \item confirmation de l’inscription,
    \item récapitulatif des bus choisis,
    \item statut de l’événement (disponible en ligne, pas encore finalisé...),
    \item code de réservation unique,
    \item instructions de paiement.
\end{itemize}

L’opération est dite \textit{best-effort} : une erreur d’envoi n’annule pas la réservation, mais un message informatif est affiché.

\subsection{Diagramme UML du processus d’inscription}

\begin{figure}[H]
\centering
\includegraphics[width=\textwidth]{figures/uml_inscription.png}
\caption{Diagramme UML de séquence du processus d’inscription}
\label{fig:uml-inscription}
\end{figure}

\section{Autres processus de l'application}

Bien que l’inscription soit le processus central, l'application gère de nombreuses autres fonctionnalités que nous présentons brièvement.

\subsection{Création d’un événement (admin)}

Ce processus suit les étapes suivantes :

\begin{enumerate}
    \item L’administrateur saisit les informations (titre, date, capacité, statut...).
    \item \texttt{EvenementService.create\_event()} valide les données.
    \item Le \textbf{DAO} insère l’événement.
    \item L’administrateur configure les bus associés.
    \item Un \textbf{mail d’alerte} est envoyé à tous les utilisateurs inscrits (fonctionnalité F08).
\end{enumerate}

Cette fonctionnalité implique directement l’API Brevo, via une boucle d’envoi.

\subsection{Modification d’un événement}

L’administrateur peut modifier la date, la capacité, ou le statut d’un événement.
La modification suit une logique similaire :

\begin{itemize}
    \item chargement des valeurs existantes ;
    \item saisie et validation ;
    \item mise à jour via \texttt{EvenementDao.update()} ;
    \item affichage de la nouvelle configuration.
\end{itemize}

\subsection{Annulation ou suppression d’un événement}

La suppression est contrôlée :

\begin{itemize}
    \item interdiction si des réservations sont encore actives,
    \item sinon suppression SQL,
    \item possibilité future d'envoyer un mail d’annulation (fonction extensible).
\end{itemize}

\subsection{Consultation des réservations personnelles}

L’utilisateur peut consulter ses réservations, grâce à :

\begin{center}
\texttt{ReservationDao.find\_by\_user()}
\end{center}

La vue affiche :

\begin{itemize}
    \item l’événement réservé,
    \item les options choisies,
    \item la date de réservation.
\end{itemize}

\subsection{Modification d’une réservation}

La fonctionnalité FO2 permet :

\begin{itemize}
    \item modification des bus choisis,
    \item modification des options (adhérent, SAM, boisson).
\end{itemize}

Un mail de confirmation de modification est envoyé à l’utilisateur.

\subsection{Suppression d’une réservation (par code)}

À l’aide d’un code unique, un administrateur peut supprimer une réservation.
Le système :

\begin{enumerate}
    \item recherche la réservation par code ;
    \item vérifie son existence ;
    \item supprime la ligne SQL ;
    \item libère les places associées.
\end{enumerate}

\subsection{Visualisation statistique}

Une vue dédiée permet à l’administrateur d’obtenir :

\begin{itemize}
    \item le nombre total d’inscrits,
    \item les taux de remplissage,
    \item le nombre d'adhérents,
    \item les taux d'utilisation des bus.
\end{itemize}

Cette fonctionnalité s’appuie sur des requêtes SQL agrégées.

\section{Synthèse}

L’application propose un ensemble cohérent de fonctionnalités couvrant entièrement les besoins du BDE :

\begin{itemize}
    \item gestion des utilisateurs,
    \item configuration des événements,
    \item gestion des bus,
    \item inscriptions,
    \item suivis en temps réel,
    \item communication automatisée par mail.
\end{itemize}

Le processus d'inscription constitue l’élément central et démontre la qualité de la décomposition en couches : chaque composant joue un rôle clair, garantissant une application robuste, maintenable et évolutive.
